\documentclass{article}

% --- PACKAGES ---
% -------------------------------------------------------------------------
%  plasTeX
% -------------------------------------------------------------------------
\newif\ifplastex
\plastexfalse


% Mathematics
\usepackage{amsmath}
\usepackage{amsthm}
\usepackage{amssymb}
\usepackage{tikz}
\usepackage{tikz-cd}

% Environments
\usepackage{graphicx}

% References
\usepackage{hyperref}
\usepackage{cleveref}

% -------------------------------------------------------------------------
%  Page formatting
% -------------------------------------------------------------------------

% Mathematical environments
\newtheorem{theorem}{Theorem}
\newtheorem{defn}[theorem]{Definition}
\newtheorem{prop}[theorem]{Proposition}
\newtheorem{lem}[theorem]{Lemma}

% -------------------------------------------------------------------------
%  Words and such
% -------------------------------------------------------------------------

% ---- LATIN ----
\newcommand*\etal{et~al.}
\newcommand*\ie{i.e.}
\newcommand*\eg{e.g.}
\newcommand*\vitae{vit\ae{}}
\newcommand*\apriori{a~priori}
\newcommand*\aposteriori{a~posteriori}  % I did not know this was a thing

% -------------------------------------------------------------------------
%  Math operators and symbols
% -------------------------------------------------------------------------

% ---- OPERATORS ----
\DeclareMathOperator{\GL}{GL}
\DeclareMathOperator{\ob}{ob}
\DeclareMathOperator{\id}{id}

% ---- SYMBOLS ----
\newcommand*\N{\mathbb{N}}
\newcommand*\Z{\mathbb{Z}}
\newcommand*\Q{\mathbb{Q}}
\newcommand*\R{\mathbb{R}}
\newcommand*\C{\mathbb{C}}
\newcommand*\F{\mathbb{F}}
\newcommand*\E{\mathbb{E}}


\title{A Group by Any \newline Other Name}
\date{2024--01--01}

\begin{document}
\maketitle

% \subsection*{What is a Group?}

% Groups are some of the ubiquitous objects in all of mathematics.
% These are structures where you have some form of addition, a ``do-nothing'' element called an \textit{identity}, and a form of subtraction acting as the opposite of addition.
% You are already familiar with groups!
% The integers $\Z$ with addition and subtraction as you know it, as well as the identity $0$ form a group.
% Invertible square matrices with real numbers as entries, denoted as $\GL(n, \R)$ for $n \times n$ matrices, form a group with addition and subtraction as multiplication by a matrix and its inverse respectively, as well as the identity matrix as the identity.
% But even here we see some structural differences.
% For any integers $x, y \in \Z$, we know that $x + y = y + x$.
% But for matrices $A, B \in \GL(n, \R)$, we may not have that $AB = BA$.
%
% But how do you define a group?
% Classically, a group is a structure with an operation obeying properties of addition and multiplication that you are familiar with, only now we will not restrict ourselves to just numbers or matrices.

Groups are one of the most common structures in all of mathematics.
They appear in many fields even outside of algebra such as differential topology and combinatorics, and have applications in areas such as chemistry and computational complexity theory.
However, what defines a group, and how much play is there in this definition?
First, let's define a group as you may have seen before.

\begin{defn}
  A group $G$ is defined as a set of elements in $G$ along with an operation $\cdot$ satisfying the following \emph{axioms}:
  \begin{enumerate}
  \item \emph{Associativity}: For any $x, y, z \in G$, we have that $(x \cdot y) \cdot z = x \cdot (y \cdot z)$;
  \item \emph{Identity}: There exists an element $e \in G$ which we call the identity such that for any $x \in G$, $e \cdot x = x = x \cdot e$;
  \item \emph{Inverse}: For any $x \in G$, there exists $x^{-1} \in G$ such that $x \cdot x^{-1} = e = x^{-1} \cdot x$.
  \end{enumerate}
\end{defn}

Writing the $\cdot$ is annoying so from now on we'll adopt the tradition of writing $x \cdot y$ as $xy$.
Similarly, we will use superscripts such as $x^{2} = xx$.

\subsection*{A Slight Relaxation}

If we take a look at the identity axiom in a group $G$, we notice that it is two-sided.
For any $x \in G$, we have that identity $e$ satisfies $ex = x = xe$.
What if we relaxed this to a right-sided version?
Define a right-handed version of the identity axiom where $e$ is the element such that for any $x \in G$, we just have that $xe = x$.
In fact, lets do the same for the inverse axiom as well.
It doesn't really make sense to do this for associativity since that's about three elements being operated on.

\begin{defn}
  A right-handed group $R$ as a set with an operation $\cdot$ (which we will still omit writing) such that
  \begin{enumerate}
  \item \emph{Associativity}: Same as before;
  \item \emph{RH-Identity}: There exists an element $e \in G$ which we call the identity such that for any $x \in G$, $xe = x$;
  \item \emph{RH-Inverse}: For any $x \in G$, there exists $x^{-1} \in G$ such that $x \cdot x^{-1} = e$.
  \end{enumerate}
\end{defn}
Clearly every group is also a right-handed group.
But is the reverse true?

\begin{theorem}
  Every right-handed group $R$ is also a group.
\end{theorem}
\begin{proof}
  We want to show that the axioms associativity, RH-identity, and RH-inverse imply the normal two-sided identity and inverse axioms.
  First we show that any element $a \in R$ satisfying $a^{2} = a$ must be the identity.
  Clearly the identity satisfies this with $e^{2} = ee = e$.
  Indeed if $a$ is such an element then it has a right-inverse $a^{-1}$.
  Thus we have that
  \[
    a = ae = a a a^{-1} = a a^{-1} = e
  \]
  and $a = e$.

  Then for any element $x \in R$, we know it has a right inverse $x^{-1}$ satisfying $x x^{-1} = e$.
  But by the above, we have $\left(x x^{-1}\right)^{2} = e^{2} = 2 = x x^{-1}$ and so $x x^{-1} x x^{-1} = \left( x x^{-1} \right) = x x^{-1}$.
  Now take $x x^{-1} x x^{-1} = x x^{-1}$ and multiply on the left by $x^{-1}$ and on the right by $x$.
  We have that
  \[
    x^{-1} x x^{-1} x x^{-1} x = x^{-1} x x^{-1} x.
  \]
  But since $x x^{-1} = e$, we may condense the left-hand side to $x^{-1} x$.
  But now we have that $x^{-1} x x^{-1} x = x^{-1} x$ and by the above we have that $x^{-1} x = e$ meaning that $x^{-1}$ is not just a right-handed inverse, but also a left-handed inverse.

  Note that nowhere above did we use that $e$ was a left-handed identity, just that $e$ is a right-handed identity.
  We now prove that $e$ is also a left-handed identity.
  Let $x \in R$.
  We know that $x$ has a (left and right)-identity $x^{-1}$ and $x^{-1} x = e = x x^{-1}$.
  Since $x e = x$, we have that
  \[
    x = x e = x x^{-1} x = e x.
  \]
  Thus $e$ is also a left-handed identity.
\end{proof}

Pretty much the same proof works if you define the analogous left-handed group and want to show that left-handed groups are also groups.
Thus we make no such distinction between groups and their left or right counterparts.
This weakening of the axioms made no difference.
However, we still had the same \textbf{number} of axioms in the definition of group and right-handed group, three axioms.
Can we define groups using two axioms?

\subsection*{Abstract Nonsense}


\end{document}
