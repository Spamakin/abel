\documentclass{article}

\usepackage[a4paper]{geometry}
\usepackage{hyperref}

\title{commutative.group}

\begin{document}
\maketitle
\newline
\begin{center}
  A Blog on Math, Algorithms, and Symmetry by Anakin Dey
\end{center}
\begin{table}
  \centering
  \begin{tabular}{cc}
    \LARGE[\href{About.html}{About}] & \LARGE[\href{All-Posts.html}{All}]
  \end{tabular}
\end{table}

% \begin{table}
%   \centering
%   \begin{tabular}{|l|}
%     \hline
%     \large 2023-12-28 \\
%     \LARGE \href{posts/abel.html}{How It's Made: commutative.group} \\
%     Everyone does ``Hello World'' or ``The Making of my Blog'' as their first post. I chose to do it as my second.
%     \hline
%   \end{tabular}
% \end{table}

\begin{table}
  \centering
  \begin{tabular}{|l|}
    \hline
    \large 2023-12-28 \\
    \LARGE \href{posts/other_name.html}{A Group by Any Other Name} \\
    As is tradition, let $G$ be a group. But, can we define what a group is in a more compact way? \\
    \hline
  \end{tabular}
\end{table}

\part*{\centering About}

\newline

My name is Anakin Dey.
I am a senior studying mathematics at the University of Illinois Urbana-Champaign.
As evidenced by this blog, my main interests in mathematics and computer science are on the algebraic and algorithmic side.
You can find more about me at \href{https://www.anakin-dey.com/}{my website}.

\part*{\centering All Posts}

\newline

Here is a list of all posts to date in reverse chronological order:

\begin{itemize}
\item 2023-12-28: \href{posts/other_name.html}{A Group by Any Other Name}
\end{itemize}

\part*{404}

\newline

Page not found!

\end{document}
