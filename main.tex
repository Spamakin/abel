\documentclass{article}

\usepackage[a4paper]{geometry}
\usepackage{hyperref}

\title{commutative.group}

\begin{document}
\maketitle
\newline
\begin{center}
  A Blog on Math, Algorithms, and Symmetry by Anakin Dey
\end{center}
\begin{table}
  \centering
  \begin{tabular}{cc}
    \LARGE[\href{About.html}{About}] & \LARGE[\href{All-Posts.html}{All}]
  \end{tabular}
\end{table}

\begin{table}
  \centering
  \begin{tabular}{|l|}
    \hline
    \large 2023-12-28 \\
    \LARGE \href{posts/other_name.html}{A Group by Any Other Name} \\
    As is tradition, let $G$ be a group. But, can we define what a group is in more interesting ways? \\
    \hline
  \end{tabular}
\end{table}

\part*{\centering About}

\newline

My name is Anakin Dey.
I am a senior studying mathematics at the University of Illinois Urbana-Champaign.
As evidenced by this blog, my main interests in mathematics and computer science are on the algebraic and algorithmic side.
You can find more about me at \href{https://www.anakin-dey.com/}{my website}.
The source code for building this site can be found on \href{https://github.com/Spamakin/abel}{GitHub}.
If you have any errors, you can email me at the email on my website or raise an issue on GitHub.
All blog posts are viewable not just as webpages, but also as PDFs if that is your thing.
Simply add \texttt{.pdf} to the url / replace \texttt{.html} with \texttt{.pdf} in the browser.

\part*{\centering All Posts}

\newline

Here is a list of all posts to date in reverse chronological order:

\begin{itemize}
\item 2023-12-28: \href{posts/other_name.html}{A Group by Any Other Name}
\end{itemize}

\part*{404}

\newline

Page not found!

\end{document}
